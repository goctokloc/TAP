\section{Dyskusja na temat jakości przybliżenia liniowego}
W przedstawionych w poprzednim podpunkcie wykresach można łatwo zauważyć, że jakość przybliżenia liniowego jest zależna od odległości od punktu pracy. Gdy odległość ta nie jest zbyt duża przybliżenie dostarcza w miarę sensowne rezultaty, mogące mieć późniejsze zastosowania. Jednakże, jeśli zadane wejście oddali się znacząco od punktu pracy wyniki bardzo rozbiegają się od modelu nieliniowego, można stwierdzić, że model liniowy nie nadaje się do ich sensownego przybliżania.

% wez tez napisz dwa slowa ze te wykresy zostaly zrobione na podstawie przeklepanych z reki dyskretnych rownan stanu do matlaba


