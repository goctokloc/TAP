\section{Symulacja działania obiektu w Matlabie}

\subsection{Otrzymany model}
Opsiany jest następującywmi równaniami

  \[  \left\{ \begin{array}{ll}
 \frac{dV}{dt}=F_H + F_C + F_D - F(h)\\
V\frac{dT}{dt}=F_H \cdot T_H + F_C \cdot T_C + F_D \cdot T_D -(F_H + F_C + F_D)\cdot T\\
F(h)=\alpha \sqrt{h}, V(h)=C\cdot h^2, T_{out}(t)=T(t-\tau), F_C(t)=F_{Cin}(t-\tau_c)
\end{array} \right. \]
\\
gdzie:
\newline
C=0,3\\
\(\alpha=9\)\\
\newline Punkt pracy zadanego układu:
\(T_C=20\degree C\)\\
\(T_H=65\degree C\)\\
\(T_D=30\degree C\)\\
\(F_C=31cm^3/s\)\\
\(F_H=20cm^3/s\)\\
\(F_D=10cm^3/s\)\\
\(\tau_c=100 s\)\\
\(\tau=40 s\)\\
\(\tau_c=45,94 cm\)\\
\(T=36,39\degree C\)\\
\newline Wielkości regulowane: h, \(T_out\)
\newline Wielkości sterujące: \(F_H, F_Cin\)\\
\newline Z otrzymanych danych wynika, że:\\
%MOZE DAJ BOLDY BY WIDAC ZE TO WEKTORY a moze i gorzej to wyglada nwm - jest git

\newline \(\textbf{x}=\begin{bmatrix}
V\\
T 
\end{bmatrix}=\begin{bmatrix}
x_1\\
x_2
\end{bmatrix}\), \(\textbf{u}=\begin{bmatrix}
F_H\\
F_{Cin} 
\end{bmatrix}=\begin{bmatrix}
u_1\\
u_2
\end{bmatrix}\), \(\textbf{v}=F_d=v_1\), \(\textbf{y}=\begin{bmatrix}
h\\
T_out 
\end{bmatrix}=\begin{bmatrix}
y_1\\
y_2
\end{bmatrix}\)




